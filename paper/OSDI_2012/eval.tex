\section{Evaluation}
\label{sec:evaluation}

\subsection{Basic Operation}
\label{sec:basic_operation}


\subsection{Pipelining \& Parallelism}
\label{sec:pipelining}


\subsection{Effect of Ballooning}
\label{sec:ballooining}

Research questions
\begin{smitemize}
\item{How much does it going to reduce overlay size?}
\item{How far does the ballooning will degrade application performance?}
\item{Does the Base VM have to be the status where ballooning is 
applied to get ballooning effect on overlay VM?}
\end{smitemize}
Memory Ballooning is the technique to control the physicall memory size of the
virtual machine. In many cases in enterprise level, memory is the most valuable 
resource in maintaining multiple VMs~\cite{} and administrator can control
VMs' allocated memory in a dynmaic way using Ballooining technique.
Usually a small balloon module is loaded into guest OS as a device driver
or kernel service, and communicate with host OS to inflate or deflate control
physical memory with the VM. It is supported by almost hypervisor including
VMWareESX, Xen and KVM, and we're goin to investigate ballooning to reduce
overlay size. Overlay VM is composed of disk image and memory snapshot as
we mentioned before and these are generated from the delta calculation between
base VM and user's custom VM. So, we can intuitively expect that overlay
size memory snapshot will be recude as the base VM and custom VM is decreased.

\paragraph{Overlay Size:}~Linux C++ application based
on the CMU MOPED object recognition libraries~\cite{MOPED2011}. 
It returns the identities of recognized objects in an input image.


